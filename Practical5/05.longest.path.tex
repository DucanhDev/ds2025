\documentclass{article}
\usepackage{graphicx}

\title{Longest Path Using MapReduce in C++}
\author{Trinh Duc Anh}
\date{\today}

\begin{document}

\maketitle

\section{Introduction}
In this practical work, we implement the **Longest Path** problem using a simple custom MapReduce framework in C++. The task is to find the longest file path(s) from a list of file paths in a distributed system. This task is a toy example of a typical MapReduce problem.

\section{Why I Chose This Implementation}
While there are established MapReduce frameworks like Hadoop and Apache Spark, implementing the problem in C++ using a custom framework allows for a deeper understanding of how MapReduce works and gives more control over performance and resource usage. C++ is a good choice for performance-sensitive tasks like this.

\section{How the Mapper and Reducer Work}
\subsection{Mapper}
The Mapper reads each file path, counts the number of directories in the path (i.e., the number of slashes `/`), and produces key-value pairs where:
\begin{itemize}
    \item \textbf{Key}: The number of directories in the path.
    \item \textbf{Value}: The full file path.
\end{itemize}

\begin{figure}[h!]
\centering
\includegraphics[width=0.6\textwidth]{mapper_design.png}
\caption{Mapper Design}
\end{figure}

\subsection{Reducer}
The Reducer receives the key-value pairs, aggregates the results, and finds the longest file path by comparing the lengths of paths. It outputs the path(s) with the maximum length.

\begin{figure}[h!]
\centering
\includegraphics[width=0.6\textwidth]{reducer_design.png}
\caption{Reducer Design}
\end{figure}

\section{System Organization}
The system consists of two main components:
\begin{itemize}
    \item \textbf{Mapper:} Processes each file path and produces key-value pairs.
    \item \textbf{Reducer:} Finds the longest file path by comparing the length of each path.
\end{itemize}

\begin{figure}[h!]
\centering
\includegraphics[width=0.6\textwidth]{system_organization.png}
\caption{System Organization}
\end{figure}

\section{Implementation Details}
The Mapper splits the file path by slashes `/` to count the number of directories in each path. The Reducer then aggregates these counts and identifies the longest path.

\begin{verbatim}
    // Mapper code snippet here
    // Reducer code snippet here
\end{verbatim}

\section{Conclusion}
This project demonstrates a custom implementation of the Longest Path problem using the MapReduce paradigm in C++. The system processes a list of file paths, counts the directories, and finds the longest file path efficiently. This approach illustrates the power of MapReduce for parallel processing in real-world scenarios.

\end{document}
