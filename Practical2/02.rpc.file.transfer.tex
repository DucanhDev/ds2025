\documentclass{article}
\usepackage{graphicx}

\title{RPC File Transfer System}
\author{Your Name}
\date{\today}

\begin{document}

\maketitle

\section{Introduction}
In this practical work, we aim to upgrade the previously implemented TCP file transfer system into an RPC-based file transfer system using gRPC. The system consists of two main components: a server that handles file upload and download requests, and a client that interacts with the server.

\section{Design of RPC Service}
The file transfer system is designed using gRPC. We define a service called \texttt{FileTransfer} with two RPC methods: \texttt{UploadFile} and \texttt{DownloadFile}. These methods allow the client to upload and download files to/from the server. The communication is based on Protocol Buffers (protobuf) for efficient data serialization.

\begin{figure}[h!]
\centering
\includegraphics[width=0.6\textwidth]{rpc_service_design.png}
\caption{RPC Service Design}
\end{figure}

\section{System Organization}
The system consists of the following components:
\begin{itemize}
    \item \textbf{Server:} Implements the gRPC service to handle file transfer requests.
    \item \textbf{Client:} Sends requests to upload or download files using gRPC.
\end{itemize}

\begin{figure}[h!]
\centering
\includegraphics[width=0.6\textwidth]{rpc_system_organization.png}
\caption{System Organization}
\end{figure}

\section{File Transfer Implementation}
The file transfer functionality is implemented using gRPC in both the client and server. Below is the code snippet for the server that handles file upload and download:

\begin{verbatim}
    // Server code snippet here
\end{verbatim}

\section{Conclusion}
This project demonstrates how to upgrade a traditional TCP file transfer system to a more modern and efficient RPC-based system using gRPC. The system allows for seamless file transfers between the client and server over a network.

\end{document}
