\documentclass{article}
\usepackage{graphicx}

\title{Word Count Using MapReduce in C++}
\author{Trinh Duc Anh}
\date{\today}

\begin{document}

\maketitle

\section{Introduction}
In this practical work, we implement the Word Count algorithm using a simple custom MapReduce framework in C++. MapReduce is a popular parallel computing paradigm used to process large datasets. Here, we simulate the process of word counting in a text file using a custom implementation of the Map and Reduce phases.

\section{Why I Chose This Implementation}
While there are existing MapReduce frameworks such as Hadoop and Apache Spark, implementing a custom framework in C++ allows for more control over the process and a deeper understanding of how the MapReduce paradigm works. Additionally, C++ provides high performance and fine-grained memory management, making it suitable for this example.

\section{How the Mapper and Reducer Work}
\subsection{Mapper}
The Mapper reads the input text line by line, tokenizes each line into words, and produces a key-value pair where the key is the word, and the value is the count (which is `1` for each word).

\begin{figure}[h!]
\centering
\includegraphics[width=0.6\textwidth]{mapper_design.png}
\caption{Mapper Design}
\end{figure}

\subsection{Reducer}
The Reducer aggregates the intermediate results produced by the Mapper. It takes the key-value pairs, sums the counts for each word, and produces the final word count.

\begin{figure}[h!]
\centering
\includegraphics[width=0.6\textwidth]{reducer_design.png}
\caption{Reducer Design}
\end{figure}

\section{System Organization}
The system consists of two main components:
\begin{itemize}
    \item \textbf{Mapper:} Reads the input and produces key-value pairs.
    \item \textbf{Reducer:} Aggregates the counts and outputs the results.
\end{itemize}

\begin{figure}[h!]
\centering
\includegraphics[width=0.6\textwidth]{system_organization.png}
\caption{System Organization}
\end{figure}

\section{Implementation Details}
The Mapper processes each line of the input file, splitting it into words, converting them to lowercase, and then creating key-value pairs. The Reducer aggregates the counts for each word and outputs the final count.

\begin{verbatim}
    // Mapper code snippet here
    // Reducer code snippet here
\end{verbatim}

\section{Conclusion}
This project demonstrates a custom implementation of the Word Count problem using the MapReduce paradigm in C++. The system processes input text, maps words to key-value pairs, and reduces them to produce word counts. This implementation provides a simple and efficient way to perform word count using parallel processing principles.

\end{document}
